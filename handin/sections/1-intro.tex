\section{Introduction}


This paper presents an analysis of the chemical composition of 12 meteorite samples, each measured across nine key components, including metal oxides, pure metals, and carbon. Using principal component analysis (PCA), the study identifies the main sources of variation in the dataset and uncovers relationships between components, such as the opposing patterns between metal oxides and pure metals and the relatively independent role of carbon. Subcompositional analysis focuses on selected subsets of components to explore detailed interactions, while self-organizing maps (SOM) are applied to detect clustering patterns among the samples. In addition, analysis of variance (ANOVA) is used to assess which compositional balances contribute significantly to differences across grouped categories. Together, these methods provide a detailed understanding of the main chemical trends, relationships, and variation patterns present in the meteorite compositions.



