\section{Conclusion}

This assignment provided a comprehensive analysis of meteorite compositional data, aiming to understand the underlying chemical patterns and their relationship to chondrite classification. Beginning with 12 meteorite samples characterized by nine components. 

Exploratory data analysis, including stacked bar plots and clustermaps, revealed clear differences in dominant components across samples, particularly highlighting the consistent abundance of $SiO_2$, FeO, and MgO. Principal Component Analysis (PCA) further identified strong correlations and opposing patterns between metal oxides and pure metals, while positioning carbon as largely independent of these trends. Subcompositional analysis revealed that variations in MgO are closely and proportionally linked to variations in $SiO_2$ and C, meaning their relative contributions remain nearly stable — the system's main variability is dominated by a single underlying pattern (PC1), while the balance between these components (captured by PC2) changes very little across the dataset.
 
Finally, a focused assessment using ANOVA and balance analysis identified which compositional components significantly contribute to differences between chondrite types. The results showed that oxides and carbon have the strongest discriminative power between groups, particularly distinguishing the carbonaceous and high-metal chondrites, while metals showed a more moderate effect.

Overall, this assignment demonstrates the value of combining compositional data techniques, dimension reduction, clustering methods, and statistical testing to uncover meaningful patterns in geochemical datasets, providing insights into both the structure and classification of meteorites.
