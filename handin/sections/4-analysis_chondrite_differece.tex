
\section{Assessment of Significant Contributors to Chondrite Type Differences using ANOVA}

The goal of this section is to assess which parts contribute significantly to the differences between the chondrite types. As indicated in Table~\ref{tab:chondrite-types}, the chondrites are characterized by the content of overall groups such as metals, carbons, and oxides. Consequently, the parts selected for significant impact analysis are based on an amalgamated version of the original components.

Specifically, the components are grouped into three categories: Oxides, Metals, and Carbon. To construct this supercomposition, the original nine unclosed parts were amalgamated and subsequently closed to sum to 100.

To avoid directly solving the linear model:

\[
\hat{x} = \beta_1 \oplus \left[ I(z = 2) \cdot \beta_2 \right] \oplus \left[ I(z = 3) \cdot \beta_3 \right]
\]

\noindent where:
\begin{itemize}
    \item \( \beta_1 \): baseline composition (e.g., for group 1, cc)
    \item \( \beta_2 \): perturbation for group 2 (hc)
    \item \( \beta_3 \): perturbation for group 3 (lc)
    \item \( I(z = k) \): indicator function (1 if \( z = k \), 0 otherwise)
\end{itemize}

the composition was first transformed using the isometric log-ratio (ILR) transformation based on the following binary partition table:

\begin{table}[h!]
\centering
\begin{tabular}{lccc}
\textbf{Balance} & Oxides & Metals & Carbon \\
\midrule
$v_1$  & 1  & 1  & -1 \\
$v_2$  & 1  & -1 & 0  \\
\end{tabular}
\end{table}

The beta coefficients for each part were then estimated using ordinary least squares (OLS). These ILR coordinates were subsequently transformed back to centered log-ratio (CLR) coordinates for interpretation:

\begin{table}[H]
\centering
\caption{CLR Beta Coefficients for hc vs cc and lc vs cc}
\begin{tabular}{lccc}
\toprule
\textbf{Contrast} & Oxides & Metals & Carbon \\
\midrule
$CLR(\beta_{hc \, vs \, cc})$ & -0.229 & 2.078 & -1.849 \\
$CLR(\beta_{lc \, vs \, cc})$ & -0.330 & 0.908 & -0.577 \\
\bottomrule
\end{tabular}
\end{table}

To further explore the \textbf{balances between parts showing similar effects} based on the previously obtained $CLR(\beta_2)$ (i.e., the contrast between \texttt{lc} vs \texttt{cc}), a new set of binary partitions was defined:

\begin{table}[H]
\centering
\caption{New Binary Partition Balances}
\begin{tabular}{lccc}
\toprule
\textbf{Balance} & Oxides & Metals & Carbon \\
\midrule
$v_1$ & 1 & -1 & 1 \\
$v_2$ & -1 & 0 & 1 \\
\bottomrule
\end{tabular}
\end{table}

The contribution of these balances was analyzed by applying an ILR transformation using the new balances, followed by OLS regression on each ILR coordinate, using chondrite type (\texttt{cc}, \texttt{hc}, \texttt{lc}) as the covariate.

For each balance $v_i$, the OLS model is given by:

\[
ILR(v_i) = \beta_0 + \beta_{hc} \cdot I(\text{hc}) + \beta_{lc} \cdot I(\text{lc}) + \varepsilon
\]

where:
\begin{itemize}
    \item $\beta_0$: intercept (representing the baseline for \texttt{cc})
    \item $\beta_{hc}$: contrast (effect) of \texttt{hc} compared to \texttt{cc}
    \item $\beta_{lc}$: contrast (effect) of \texttt{lc} compared to \texttt{cc}
    \item $I(\text{hc})$ and $I(\text{lc})$: indicator functions (1 if the sample belongs to that type, 0 otherwise)
\end{itemize}

The significance of these effects was evaluated using the \( t \)-values and \( p \)-values from the model, as shown in Table~\ref{tab:anova_ilr_results}. As expected from the type definitions, oxides and carbon exhibited statistically significant effects in distinguishing the chondrite types, particularly between the carbonaceous (cc) and high-metal (hc) groups, as indicated by the significant \( p \)-value (0.0159) in the Oxides+Carbon balance.

Additionally, the metals balance displayed a moderately significant difference (\( p = 0.0454 \)) between the groups, primarily driven by the increased metal content in hc relative to cc.

These findings indicate that both the oxide/carbon composition and the metal content are key contributors to the chemical differentiation between chondrite types, with oxides and carbon exerting the strongest discriminative effect.

\begin{table}[H]
\centering
\caption{ANOVA Results per ILR Balance}
\label{tab:anova_ilr_results}
\begin{tabular}{lcccc}
\toprule
\textbf{Balance} & \textbf{Intercept (cc)} & \textbf{Beta hc vs cc} & \textbf{Beta lc vs cc} & \textbf{$t$-value / $p$-value} \\
\midrule
Oxides+Carbon & 0.8793 & -2.5447 & -1.1119 & 2.6074 / 0.0159 \\
Metals & 3.7722 & 1.1458  & 0.1746  & 2.1083 / 0.0454 \\
\bottomrule
\end{tabular}
\end{table}
